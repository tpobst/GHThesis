\chapter{Introduction} \label{ch:intro}

\section{Description of Study Area}

\begin{figure}[h!]
  \centering
  \includegraphics[width=6in]{location}\\
  \caption{Location of the Great Smoky Mountain National Park}\label{fig:location}
\end{figure}

The Great Smoky Mountains National Park (GRSM), which is located in the southern Appalachians spanning eastern Tennessee and western North Carolina, is the second largest national park in the eastern United States.
It contains roughly 100 species of native trees, more than 1,500 flowering plants, 200 species of birds, 66 types of mammals, 50 native fishes, 39 kinds of reptiles, and 43 species of amphibians.
The unique nature of the park has earned it the title of  International Biosphere Reserve by the United Nations \citep{NPS}.    
Partly due to this diversity of life, the GRSM is one the most visited parks in the U.S. and its conservation is a high priority for the National Park Services (NPS) which is tasked with preserving it.        
Park conservation is constantly changing and  includes monitoring streams for the consequences of acid deposition.     
    
\section{Acid Deposition and the GRSM}

Acid deposition is characterized as wet deposition (rain and snow), dry deposition (gases and particles), and fog or cloud deposition (occult) \citep{lovett1994atmospheric}.
These three modes of pollutant transportation, transport and deposit acid pollution over the industrialized world \citep{board1983acid}.
The dominant man-made source of  acid deposition is fossil fuel combustion, such as gas engines for transportation, and industrial plants for production and power \citep{EPA2}.
Power plants expel sulfur oxides (SO$_x$) and nitrogen oxides (NO$_x$) high into the atmosphere through smoke stacks where they react and fall to the earth as acid deposition.
Once the pollutants have entered the environment they react with hydroxide and oxygen in the air, the surface waters, the soil, and on man-made structures to produce acids \citep{board1983acid}. 
 
The upper elevations of the GRSM receive some of the highest loading rates of acidifying nitrogen and sulfur species in North America \citep{johnson1992atmospheric, weathers2006}.  
Acid deposition will acidify surface waters, potentially harming aquatic biota \citep{neff2009physiological}.
The rate of stream acidification is related to both the concentration of ANC (acid neutralizing capacity) and base cations present in the stream.
ANC can decrease along with stream pH partly due to base cation depletion, which can occur through leaching, another cause of increased acid input.
In this process, the inherent base cation minerals react and run out leaving excess H$^+$ and Al to be released into the water \citep{sullivan2004}.
The increased H$^+$ concentration will lower the pH and the presence of Al can be toxic to fish \citep{driscoll2003effects}.
A constant removal of base cations can lead to permanent ANC values less than zero or chronic acidification.

Acidification of water bodies can be either chronic or episodic. 
Chronic acidification occurs when the body of water has constant low ANC, which creates a large area of  almost un-inhabitable water where aquatic life would struggle to survive. 
Episodic acidification describes a rapid increase of acidity due to large surges of pollutants, usually from snow melts or heavy rains.  
While chronic acidification may inhibit habitation, episodic acidification can kill aquatic life by quickly dropping the pH of streams, not allowing for adaptation or moving \citep{sullivan2004}.
A literature review in \citet{neff2009physiological} approximates a pH of 6 for negative biological effects and a pH of 5 for mortality for trout in the park.  
Stream pH levels between 5 and 6 can become toxic in the presence of aluminum through leaching and base cation exchange.
This toxicity can be harmful to eggs and fry in very soft waters in the lower end of the range.  


\section{The Park-Wide Stream Survey} 

The park-wide stream survey began as part of the park's Inventory and Monitoring program of the GRSM in 1993 in response to acidification of the parks streams \citep{harwell2001}.
Currently grab samples are collected six times per year from 32 sites in five watersheds (Abrams, Cataloochee, Cosby, Little, Oconaluftee) and twice a year from 11 sites in Hazel Creek in order to monitor the health of the park's streams.
Historically there were nearly 390 sites listed in the stream survey, but over time the number of sites monitored has been reduced to the 43 sites used in this study.
Every sample is  measured for pH, ANC, conductivity, acid anions (Cl$^-$, SO$_4^{2-}$, NO$_3^-$, ammonium (NH$_4^+$)), the base cations (Ca$^{2+}$, Mg$^{2+}$, K$^+$, Na$^+$), and dissolved metals (Al, Cu, Fe, Mn, Si and Zn).  
A ManTech$^{TM}$ autotitrator was used for pH, ANC, and conductivity.  
A Dionex$^{TM}$ ion chromatograph (IC) was used for the analysis of Cl$^-$, SO$_4^{2-}$, NO$_3^-$, and NH$_4^+$.  
A Thermo-Scientific$^{TM}$ Inductively Coupled Plasma - Atomic Emission Spectrometry (ICP-AES) was used for the study of Ca$^{2+}$, Mg$^{2+}$, Na$^+$, K$^+$, Al, Cu, Fe, Mn, Si and Zn.

\begin{figure}[h!]
  \centering
  \includegraphics[width=6in]{SSsites}\\
  \caption{Site locations for the Stream Survey from 1993 to 2009. }\label{fig:SSsites}
\end{figure}

All data is collected, under contract, for the NPS.
Study site identifiers such as time, place, water temperature, and weather are recorded on site for each sample.
This data goes all the way back to the beginning of the survey in 1993.
Along with specific sample measurements, each sample is labeled by its site ID, which indicates the location.
Several important characteristics are known for each site, such as stream name, geology, and elevation.
All of these are used to study acid deposition in the GRSM.
In 2003, the survey was improved by carefully decreasing the number of  sites from 90 to the current 43 \citep{odom2003}.
Discontinuation of sites periodically throughout the years has created a non-uniform database making statistical analysis problematic.
Along with different sites for different years, the base flow/storm flow classification is also inconsistent.
The data is labeled base flow/storm flow beginning in 1993 and ending in 2010, while data up to 2012 will be analyzed here.
Two years of sample data are missing this classification.

\subsection{Elevation Bands}

Elevation and basin location, are some of the more important characteristics governing water chemistry of each site location\citep{neff2012influence}.
Elevation was found to be a dominant driver for predicting water quality among the park's streams. 
Many of the water quality variables can be characterized by elevation: pH, ANC, the acid anions, and the base cations.
Overall, results from the Biotics Effects report found that stream pH and ANC decreased at -.32 units and -35.73 $\mu eq L^{-1}$ respectively, per 1,000-ft elevation gain \citep{cai2013}. 
Many factors affect the pH of mountain streams, but clouds affect higher elevations more than they affect lower elevations, sometimes accounting for lower pH values in the higher elevations \citep{shubzda1995}.
The correlation between pH and elevation is studied in the park by the use elevation bands.
Conductivity, Cl$^-$, and base cations were also found to be correlated to elevation by significantly decreasing as elevation increases.  
SO$_4^{-2}$ showed no significant trend with elevation, however NO$_3^-$ was found to significantly increase with elevation gain.  
The GRSM 2011 Annual Water Quality Report compared pH trend lines representing the current 43 sites from 1993 through 2010 \citep{annualreport2012}.  
The data showed lines of similar slopes with different intercepts, which was interpreted to mean increasing pH at all elevations in GRSM streams.  
The elevation bands are also useful because higher elevations experience increased SO$_4^{-2}$ and prolonged acidification if soil desorption becomes a dominant geochemical watershed process, which could occur if pH increased to 6.0 and SO$_4^{-2}$ dropped below 50 μeq L-1 \citep{annualreport2012}.  
From a management perspective, \citet{cai2013} describes potential limitations in the data to assess long-term changes in stream water quality because locations sampled have changed over time and most of the current sample locations are at lower elevations.

\begin{table}[htbp]
\centering
\begin{tabular}{ccccc}
\toprule
Elevation class & \multicolumn{1}{p{3cm}}{Range of elevation (ft) MSL} & \multicolumn{1}{p{3cm}}{Number of sampling sites} & \multicolumn{1}{p{2cm}}{Percent of NPS area*} & \multicolumn{1}{p{2cm}}{Percent of sampling sites} \\ 
\midrule
1 & $<$1000 & 0 &  & \\ 
2 & 1000-1500 & 7 & &  \\ 
3 & 1500-2000 & 13 &43.3 &65.0  \\ 
4 & 2000-2500 & 16 &  &  \\ 
5 & 2500-3000 & 18 & &  \\ \midrule
6 & 3000-3500 & 13 & 27.4 & 20.5 \\ 
7 & 3500-4000 & 4 & & \\ \midrule
8 & 4000-4500 & 5 & 21.2 & 12.1 \\ 
9 & 4500-5000 & 5 & &  \\ \midrule
10 & 5000-5500 & 1 &8.1 & 2.4 \\ 
11 & $>$5500 & 1 & &  \\ 
\bottomrule
\end{tabular}
\caption{Historical elevation bands for the 90 site survey. *Approximate percentages based on planimetering contour map}
\label{tab:Odomtable}
\end{table}


\begin{table}[htbp]

\begin{tabular}{cccc}
\toprule
Elevation class & Range of Elevation m(ft) & \multicolumn{1}{p{3cm}}{Number of sampling sites} &\multicolumn{1}{p{3cm}}{Percent of sampling sites} \\ 
\midrule
1 & $<$304.8 ($<$1000) & 0 &  \\ 
2 & 304.8-457.2 (1000-1500) & 4 &  \\ 
3 & 457.2-609.6 (1500-2000) & 4 & 67.4 \\ 
4 & 609.6-762 (2000-2500) & 9 &  \\ 
5 & 762-914.4 (2500-3000) & 12 & \\ 
\midrule
6 & 914.4-1066.8 (3000-3500) & 6 & 16.3 \\ 
7 & 1066.8-1219.2 (3500-4000) & 1 &  \\ 
\midrule
8 & 1219.2-1371.6 (4000-4500) & 3 & 14.0\\ 
9 & 1371.6-1524 (4500-5000) & 3 &  \\ 
\midrule
10 & $>$1524 ($>$5000) & 1 & 2.3\\ 
\bottomrule
\end{tabular}
\caption{Historical elevation bands for the 43 site survey}
\label{tab:43sitesurvey}
\end{table}


\autoref{tab:Odomtable} represents the poorly distributed sites within elevation bands presented by Kenith Odom as table 38 in \citet{odom2003}.
His dissertation suggested a remodel of the survey from 90 sites to 43 and this table was used to suggest more high elevation sites.
The survey was reduced from 90 sites down to 43 but the elevational distribution was not fixed.
For comparison \autoref{tab:43sitesurvey} shows the percentage of sites per elevation bands for the 43 site survey, just as \autoref{tab:Odomtable} does for the 90 site survey.
There are very few differences between the percentages, meaning there are still problems with the elevational distribution.

\begin{figure}[h!]
  \centering
  \includegraphics[width=6in]{DepositionHeat}\\
  \caption{ Modeled atmospheric deposition of N and S for the year 2000 and presented in \citet{weathers2006}.}\label{fig:depositionheat}
\end{figure}

For overall acidification of the GRSM, the high elevation bands could be the most important in the survey but they have the least amount of representation \citep{cai2013}.
As can be seen in \autoref{fig:depositionheat}, which is a model, the highest deposition of sulfur and nitrogen is at the highest elevations.
This is because rainfall and fog in the GRSM affect elevations above 4000 feet first and higher elevations have steeper slopes, which correlate to both thinner soils and base poor geology.  
Sites in these areas continue to receive low pH values in samples and representation at these elevations is important.
Unfortunately the tenth elevation class, according to \autoref{tab:43sitesurvey}, has only one site in it and one site cannot represent a whole elevation band.

Without adding sites, the easiest way to fix this poor distribution is to reorganize the elevation bands. 
For this paper the elevation bands were reassessed to strengthen the higher elevations.
%A cluster analysis was explored for the task but it was not successful.
%There was too much variation to cluster by elevation only.
%A short report of the cluster analysis is presented in \hyperref[ch:CA]{Appendix H}.
%Because the cluster analysis was not successful, the elevation boundaries which divided the bands were moved to include more or less sites.

\begin{table}[htbp]
\caption{Elevation Bands}
\begin{tabular}{clcp{5cm}}
\toprule
Elevation Bands & Meters (Feet)                              & n & Site \# \\ 
\midrule
1                        & 304.8-609.6 (1000-2000)           & 5   & 13 ,23, 24, 30, 479 \\ 
2                        & 609.6-762 (2000-2500)              & 9   & 4, 311, 268, 480, 310, 483, 147, 148, 484 \\ 
3                        & 762-914.4 (2500-3000)              & 13 & 114, 481, 482, 149, 66, 492, 137, 293, 270, 493, 485, 144, 224 \\ 
4                        & 914.4-1066.8 (3000-3500)         & 4   & 143, 142, 73, 71 \\ 
5                        & 1066.8-1371.6 (3500-4500)       & 4   & 74, 221, 251, 233 \\ 
6                        & $1371.6< (4500<)$                    & 2   & 253, 234 \\ 
\bottomrule
\end{tabular}
\label{tab:ElevationBands}
\end{table}

\autoref{tab:ElevationBands} contains all the sites used in the statistical analysis.
Each of the statistical analyses in this paper will use these elevation bands to classify elevation for the stream survey data.

\section{Data Analysis Time Periods}

Time trends are a general way to assess changing conditions of stream health in the GRSM.
Instead of representing a single point in in time like each grab sample, the trend analysis represents a site over time.
The analysis can be used for the current quality of the streams in the survey along with trends to determine where the quality is headed. %
Trend analyses were conducted on the stream survey data in 2002 and published in \citet{robinson2008ph} and then again in 2009 for the Biotics Effects report \citep{cai2013}.
Though these papers analyzed similar years, \citet{robinson2008ph}:1993-2002 and \citet{cai2013}: 1993-2009, the results of these analyses are in disagreement.
Of the ten elevation bands analyzed in \citet{robinson2008ph} six had negative Julian date coefficients and the other four had no trend.
The conclusion was reached that the pH is headed towards harmful and lethal conditions for aquatic life. 
In \citet{cai2013}, of the 67 sites studied in the biotic effects report most showed no trend, 22 showed an increase in pH and only 2 showed a decrease. 

The opposite trends reported in  \citet{robinson2008ph} and \citet{cai2013} suggest an inflection point in the trend line somewhere between 2002 and 2009. 
For this reason, and for easier comparison of results,  a separate data set will be partitioned off from 1993 to 2002 to equal the years analyzed in \citet{robinson2008ph}.  
A third data set will be partitioned after the year 2008 because this is the year that the Kingston and Bull Run coal-fired power plants installed scrubbers onto their smoke stack exhaust. 
The hypothesis being the SO$_4^{2-}$ concentrations will be noticeably different, and this difference could indicate greater SO$_4^{-2}$ export from the soils. 
These three time sets will be analyzed separately (1993-2002, 2003-2008, 2009-2012).

\section{Data smoothing}  \label{sec:smoothing}%put seasonality back into trend analysis

It is rare for water quality data to be perfectly normal or parametric which is required for most statistical analysis.
It is usually  non-parametric and can contain recording errors and other influential values \citep{helsel1992statistical}.
Four water quality variables will be used as dependents throughout this paper: pH, ANC, NO$_3^-$, SO$_4^{2-}$.
Each of these dependents is important for studying acid deposition: pH and ANC directly relate the health of the streams, NO$_3^-$ and SO$_4^{2-}$ are the man-made pollutants thought to be causing increased acid deposition.
Before these variables can be used as dependents they need to be analyzed for distribution, outliers, cycles, missing values, and serial correlation \citep{helsel1992statistical}.
After initial analysis, all of the dependent vectors had outliers, most of these were found as a part of the step-wise regression process which highlights influential data for further analysis.

\begin{figure}[h!]
\centering
    \includegraphics[width=6in]{pHdata}\\
    \caption{pH plotted vs. Elevation. With and without outliers.}
    \label{fig:pHdata}
\end{figure}

The entire data smoothing process will not be shown here, but pH will be shown as an example.
A figure of pH vs. month clearly shows seasonality, a pattern in pH based on the seasons, which is important to address for trend analysis \citep{helsel1992statistical}.
\autoref{fig:pHdata} shows the pH vs. elevation plot, which exhibits some outliers but also a negative trend in pH as elevation increases.
This graph includes two trend lines, one which represents the trend of all of the data points and the other representing a trend after the influential points are removed. 
Both trends are negative as elevation increases but the trend line containing the influential points is steeper. 

Much of the variance in \autoref{fig:pHdata} can be attributed to known influences in the stream survey data: Abram's creek watershed, sites that are affected by anakeesta geology, and storm flow \citep{neff2012influence}.  
The anakeesta formation contains sulfidic slate, which can have the same negative effect of acid deposition,  and keeps the pH values of streams very low.
Site numbers 237 and 252 are sites that are downhill from road cuts that have exposed the underlying anakeesta formation to runoff, these sites have consistently low pH values compared to nearby sites.  
Comparatively, Abrams is a low elevation, low slope area where the underlying geology is Cades Sandstone, which buffers against acid rain extremely well. 
This sandstone contributes to high ANC values which in turn keep the mean pH levels higher than the rest of the sites in the survey. 

Storm flow is both influential and detrimental  to GRSM water quality. 
Storms can bring high intensity rainfall, quickly adding pollutants from rain, storm runoff, and pollutants left in the soil.
Adding pollutants to streams with already low ANC and pH can be very harmful to aquatic life.
Along with measured ANC, measurements taken from storm flow can show uncharacteristically low pH values and high amounts of metals from leaching. 
In this way, storm flow is sometimes considered an influential group on the rest of the data. 
Dr. Cai characterized all of the available water quality data between 1993 and 2010 as storm flow or base flow; this work is summarized in \citet{cai2013}. 
Because this classification ends at data collected in 2010, classifications for the years 2011 and 2012 would need to be tabulated for this paper.
Quick analyses were run to see how influential storm flow was on the data as a whole, but results were inconclusive. 
Instead of dismissing all of the storm flow observations at once, single influential observations could be explained by storm flow and removed. 
These observations can be removed on a case by case basis during the regression method in \autoref{sec:swregression}.

\section{Objectives}
Guidelines for this study were to:
\begin{itemize}
\item  Characterize time trends in stream pH and acidic anions among elevation ranges in order to evaluate whether conditions are improving or degrading.
\item Characterize sampling variance based on available water quality data, within the context of time and elevation, to support development of the GRSM’s Vital Signs Monitoring Program. 
\end{itemize}
The format of this thesis will follow these two objectives.  
\begin{itemize}
\item Determine if stream pH and acid anion concentrations changed among three time periods (1993-2002, 2003-2008, and 2009-2012), and among six elevation ranges (1000-2000 ft, 2000-2500 ft, 2500-3000 ft, 3000-3500 ft, 3500-4500 ft, 4500$<$ ft).
\begin{itemize}
\item Time trends (\autoref{ch:TA})
 \item Means Comparisons (\autoref{ch:mc})
\end{itemize}
\item Determine the statistical power for water quality parameters based on frequency and elevational location.
\begin{itemize}
\item Post Hoc Analysis (\autoref{ch:poweranslysis})
\item A Priori Analysis (\autoref{ch:poweranslysis})
\end{itemize}
\end{itemize}
The thesis is organized into three separate chapters addressing the above research questions. Each chapter will follow the technical format of introduction, methods, results, and discussion. 




%vital signs
%The National Park Service is currently developing a Vital Sign Monitoring Program for a number of national parks around the country, which includes the GRSM (Annual Administrative Report For Inventories and Vital Signs Monitoring FY 2010).  
%The goal of the GRSM Vital Sign Monitoring Program is to evaluate long-term changes in ecosystem health, including both terrestrial and aquatic environments. 
%In addition, the program will integrate these environments and and include existing data on basin factors associated with water quality.   
