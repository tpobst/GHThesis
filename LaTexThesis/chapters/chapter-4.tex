\chapter{Power Analysis} \label{ch:poweranslysis}
\section{Methods}
\subsection{Introduction}
The statistics preformed in \autoref{ch:TA} come with an inherent amount of error.  
This error in trend analysis is defined as type II or $\beta$ and can be seen in \autoref{tab:HypothesisTests}.
\begin{table}[htbp]
\centering
\caption{Hypothesis tests from Statistical Methods in Water Resources by theUSGS  \citep{helsel1992statistical}.}
\begin{tabular}{l|p{4cm}|p{3.8cm}}
\toprule
                                   & H$_0$ is true                                                                                          & H$_0$0 is false                                                                                     \\
\midrule
Fail to Reject H$_0$  &Correct decision Prob(correct decision) = 1 -$\alpha$                             & Type II error Prob(Type II error) = $\beta$                                         \\ \cline{2-3} \noalign{\smallskip}
 Reject H$_0$            & Type I error Prob (Type I error) = $\alpha$ \textbf{Significance level} &Correct decision Prob (correct decision) = 1-$\beta$ \textbf{Power}  \\ 
\bottomrule
\end{tabular}
\label{tab:HypothesisTests}
\end{table}                            
The trend analysis tests the hypothesis that a trend exists in the data.
The null hypothesis is that there is no trend or the coefficient is zero.
In simpler terms this means that the hypothesis for the trend analysis is that there is a positive julian date coefficient for each dataset and the null hypothesis is that there is not one.
$\beta$ describes the failure to reject a false null hypothesis or in the case of this paper a failure to detect a trend in the data when there really is one.
The opposite of $\beta$ is the probability that a trend will be detected when it exists and is called the power of the test.
A trend line with a power of 1.00 means that there is a 100$\%$ chance that the calculated slope is not zero.
The post hoc power of all 144 regression lines from \autoref{ch:TA} were calculated with the help of the statistical program G*power.  
An a priori power analysis can be used to help plan for the future by choosing a desired power and planning around it.
This analysis was also completed in G*power.
% explain ES up here somewhere Cohen thinks of ES values as chances to find a significant result.
\subsection{Body}
The objectives of the power analysis are two fold.
It will calculate the power of each trend line and also give a more efficient sample size for desired powers.
The statistical program G*power requires four inputs total.  
Three of the these inputs are passed from the output of the trend analysis: number of observations (N), adjusted r$^2$ and number of predictors.
The fourth input is ES or effect size which is calculated by G*power before the analysis.
G*power is a free power analysis program written by four germen psychology professors.
It can compute both post hoc and a priori analysis for many different statistical tests.
A post hoc analysis was preformed for both julian date coefficient tables from \autoref{ch:TA},  \autoref{tab:SWposthoc} and \autoref{tab:TVposthoc}.
In contrast to the post hoc analysis the a priori analysis only needs to be calculated for each of the four dependent variables.
This output then has to be manipulated in excel in order to determine a number of sites per elevation bands.
%add something about calculating sites per elevation band for a desired power
\subsection{Procedures}
\subsubsection{Post hoc}

The most popular power analysis methods originate from Jacob Cohen who outlined his approach in " A Power Primer" \citep{cohen1992power}.
Cohen displayed ways to calculate the power for eight different tests the last of which is the F test for multiple and multiple partial correlation.
The different tests are represented by their differences in calculating ES.
Effect size is the only input that needs to be calculated before the analysis can be completed.
The equation for the ES of a regression model presented by Cohen is equal to the correlation coefficient divided by one minus the correlation coefficient.
This equation can be described as the ratio of explained to unexplained variation for the regression model.
For the post hoc analysis this equation will be used to calculate an specific ES for each model presented in \autoref{tab:SWposthoc} and \autoref{tab:TVposthoc} which correspond to the models in \autoref{ch:TA}.
This calculation is completed in G*power after inputing the correlation coefficient.
G*power uses ES along with the $\alpha$ used for the regression model, the number observations, and number of predictors in the model to output the power of the F test.
This power will be between 0 and 1.00 and will be the power acquired by the models using past data and a calculated ES.

\subsubsection{A priori}

The a priori analysis is more conditional than the straight forward calculations for the post hoc analysis.
Instead of outputting a power value like the post hoc analysis, G* power will output a value for the number of observations. 
The inputs for this analysis are $\alpha$, desired power, number of predictors, and ES.
All of these inputs can be changed or manipulated based on the desired outcome.
For this analysis the assumption was made that the same trend analysis as the one completed in \autoref{ch:TA} would be attempted in the future.
Based on this assumption the same step-wise equations constructed in \autoref{tab:stepwiseeq} can be used to help chose the number of predictors and $\alpha$.

The most encompassing way to present an a priori analysis is through a power graph.
The power graphs plot power on the y-axis and number of observations on the x-axis.
Using this as a tool a planner can choose a desired power and get the corresponding number of observations.

Choosing an ES value and desired power will be a matter of convention.
To make choosing the ES value easier Cohen has stated small, medium, and large ES values for each of the eight tests described in \citet{cohen1992power}.
Concerning the multiple and multiple partial correlation test he decided on .02, .15, and .35 respectively.
All of these ES values can be graphed in the power graphs by plotting different ES values as curves on the same plot.
Even though all powers and all ES values can be plotted at once, some times it is useful to choose specific values.
An ES value of .15 was settled upon after the power graphs for all three conventions per dependent variable were made.
.02 was too small, requiring very high values for numbers of observations to reach a decent power.
ES values of .35 can acquire small numbers for numbers of observations thus achieving a decent power level easier.
But .15 is less than half of .35 so it minimizes the chances for insignificant results and the numbers of observations are reasonable to reach higher powers.
If no argument can be made for any other desired power then Cohen suggests .80.
This is chosen for its reasonable ratio of Type I error to Type II error which reflects their importance.
If the power is .80 then $\beta=.20$ and $\alpha=.05$ and this makes the Type II error four times as likely as Type I error \citep{cohen1992statistical}.
These choices are presented in \autoref{tab:APN}.
\begin{table}[htbp]
		\caption[A priori results]{A priori calculation in G*power when alpha, ES, and power are set to .05, .15, and .80 respectively.}
		\begin{center}
		\begin{tabular}{lcc}
		\hline\noalign{\smallskip}
		 & \multicolumn{1}{l}{Number of predictors} & $N_a$ \\  \hline\noalign{\smallskip}
		pH & 6 & 98 \\ 
		ANC & 8 & 109 \\ 
		Nitrate & 8 & 109 \\ 
		Sulfate & 7 & 103 \\ 
		Time & 3 & 77 \\  \hline
		\end{tabular}
		\end{center}
		\label{tab:APN}
		\end{table}

The a priori power analysis can be manipulated to calculate a number of sites per elevation band for the stream survey in the GRSM.
First, samples per year per elevation band are counted for the 2012 year and will be represented by n.
Next the results from \autoref{tab:APN} are divided by samples per year per elevation band to get the number of years it will take at the 2012 sampling rate to reach a power of .80.
\begin{equation}
	yrs. = {N_a \over n}
\end{equation}
But, in order to get to the number of sites per elevation band required to reach a power of .80, the years will have to be held constant.
If the future trend analysis is to be completed using the equation with only time variables then 77 samples will need to be collected in one year to reach a power of .80 according to \autoref{tab:APN}.
But if the future trend analysis is to be completed using the step-wise equations from \autoref{tab:stepwiseeq} then at least 109 samples will need to be collected in one year to satisfy the requirements for ANC and NO$_3$.
For the step-wise equations N will be rounded up to 110 and labeled $N_b$.
\begin{table}[htbp]
\caption{samples/year to achieve a power .80}
\begin{center}
\begin{tabular}{lrrrr}
\hline\noalign{\smallskip}
Years & 1 & 2 & 3 & 4 \\ \cline{2-5}\noalign{\smallskip}
Water Quality Variables & 110 & 55 & 37 & 28 \\ 
Time Variables & 77 & 39  & 26  & 19  \\ \hline\noalign{\smallskip}
\end{tabular}
\end{center}
\label{tab:sytaapeighty}
\end{table}
These are presented in \autoref{tab:sytaapeighty}, which has been calculated out to four years.
So that instead of completing the trend analysis after one year, one could wait four years and only need to collect 28 samples per year.
Subtracting the number of samples collected in one year per elevation band in 2012 from the number of samples needed to be collected per year to reach a power of .80 will provide the number of samples needed per elevation band to receive a power of .80 ($N_c$).
\begin{equation}
	N_c={N_b - n}
\end{equation}
To get an estimation for the number of sites needed per elevation band to achieve a power of .80, the number of samples needed per elevation band to receive a power of .80 ($N_c$) were divided by six which is number of times each site is sampled per year.
\begin{equation}
	\#Sites = {N_c \over 6}
\end{equation}

\section{Results}
\subsection{Post hoc}%Most ES values are larger than .35, wtf!
The results of the post hoc analysis on both trend analyses are reported in  \autoref{tab:SWposthoc} and \autoref{tab:TVposthoc}.
The tables of results are broken into the four analyzed water quality variables (pH, ANC, NO$_3$, SO$_4$) and divided into the tree time sets (93-02, 03-08, 09-12), and then further divided into the six elevation classes.
Each regression line computed in \autoref{ch:TA} are represented by their number of observations, the adjusted $r^2$, the calculated ES, and finally their actual power.
Of the 72 lines evaluated for power in \autoref{tab:SWposthoc} only eight of them were less than 1.00.
Two of the trend lines in \autoref{tab:SWposthoc} were insignificant.
The trend line in Nitrate set 3 class 5 was insignificant and the adjusted $r^2$ was negative and therefore the power could not be found.
The trend line in pH set 3 class 5 was also insignificant and received the lowest actual power of .28.
Of the 72 lines evaluated for power in \autoref{tab:TVposthoc} 52 of the them were insignificant.
So already before the post hoc analysis is completed the trend analysis says the regression lines might not even have a trend, and power is the probability of finding a trend if there is one.
By convention insignificant trends are ignored and their powers will be ignored here as well.
20 of the trend lines are significant and their powers range from .26 to 1.00.
11 of the 20 significant trend lines are above a power of .80 and 2 are .99 or greater.

\subsection{A priori}
\subsubsection{Power graphs}
The traditional presentation for an a priori power analysis is the power graph.
This graph plots power on the y-axis and number of observations on the x-axis.
Each plotted curve represents an ES of either .15 or .35.
Four graphs were drawn for the five dependent variables (pH, ANC, NO$_3$, SO$_4$, and Time)
The only real variation between the graphs are the number of variables which are taken from \autoref{tab:stepwiseeq}.
Because the ANC and NO$_3$ equations contain the same number of variables, they are represented by the same graph.
The curve will start at (0,0) and asymptotically approach a power of 1.00.
All of the power graphs created by the a priori analysis follow this pattern.
In fact they are all very similar.
The curves are all the same, its their placement on the graphs that change.
The power graphs for pH, ANC and NO$_3$, SO$_4$, and Time are plotted in \autoref{fig:pHPowerGraph}, \autoref{fig:ANCnNPowerGraph}, \autoref{fig:SulfatePowerGraph}, and\autoref{fig:TVPowerGraph}.
\autoref{powergraphtable} was created for easier comparison.
\begin{table}[htbp]
\centering
\caption{Sample sizes at a power of .80}
\begin{tabular}{lrr}
\toprule

 ES & 0.15 & 0.35 \\ 
 \midrule
pH & 97 & 45 \\ 
ANC and NO3 & 98 & 51 \\ 
SO4 & 103 & 48 \\ 
Time & 76 & 35 \\ 
\bottomrule
\end{tabular}
\label{powergraphtable}
\end{table}

This table shows the sample size values for both ES curves at a power of .80.
Again all are similar except for the time graph, which has at least half as many variables in its time trend equation.

\subsubsection{A priori manipulation}
\begin{itemize}
	\item Using the ability of the a priori power analysis to compute a number of samples needed for a certain power, a scenario was played out to see how many sites needed to be added or could be removed from an elevation band in the stream survey.
	\begin{table}[htbp]
\caption{Years to achieve a power of .80}
\begin{tabular}{clccccc}
\toprule
\multicolumn{1}{p{1.5cm}}{Elevation Bands} & \multicolumn{1}{c}{Site \#} & \multicolumn{1}{p{1.5cm}}{Current n/yr} & \multicolumn{1}{p{1cm}}{pH} &\multicolumn{1}{p{1cm}}{ ANC NO$_3$} & SO$_4$ & \multicolumn{1}{p{2.2cm}}{Time variables} \\  
\midrule
1 & 13 ,23, 24, 30, 479 & 26 & 3.77  & 4.19  & 3.96  & 2.96  \\ 
2 & \multicolumn{1}{p{4cm}}{4, 311, 268, 480, 310, 483, 147, 148, 484} & 34 & 2.88  & 3.21  & 3.03  & 2.26  \\ 
3 & \multicolumn{ 1}{p{4cm}}{114, 481, 482, 149, 66, 492, 137, 293, 270, 493, 485, 144, 224} & \multicolumn{ 1}{c}{62} & \multicolumn{ 1}{c}{1.58 } & \multicolumn{ 1}{c}{1.76 } & \multicolumn{ 1}{c}{1.66 } & \multicolumn{ 1}{c}{1.24} \\ 
4 & 143, 142, 73, 71 & 24 & 4.08  & 4.54  & 4.29  & 3.21  \\ 
5 & 74, 221, 251, 233 & 22 & 4.45  & 4.95  & 4.68  & 3.50  \\ 
6 & 253, 234 & 12 & 8.17  & 9.08  & 8.58  & 6.42  \\  
\bottomrule
\end{tabular}
\label{tab:currentyrsto.80}
\end{table}%needs to be cleaned
	\item This scenario was followed through with both methods of trend lines.
	\item \autoref{tab:currentyrsto.80} records the six elevation bands along with the site numbers that belong to them.  In the column labeled ,current n per year, the amount of samples collected per elevation band in the year 2012 was tabulated.  The values in the remaining columns were calculated by dividing the number of samples given in \autoref{tab:APN} by the current samples per year column in \autoref{tab:currentyrsto.80}.
	\item Looking at the table there are 26  samples collected in elevation band one in one year.  In order to compute a trend line that receives a power of .80 with pH as the dependent  samples would need to be collected for 3.77 years before the trend line is computed.   The larges is elevation class for a trend line in ANC or NO$_3$ which requires 9.08 years.
	\begin{table}[htbp]
\centering
\caption{Necessary sites scenario for water quality variables}
\begin{tabular}{ccccc|ccccc}
\toprule
\multicolumn{1}{c}{} &   \multicolumn{4}{c}{ \#Samples required} & \multicolumn{4}{c}{\# sites required} \\ \cline{2-9}\noalign{\smallskip}
\multicolumn{1}{p{3cm}}{Elevation Bands}  & 1 yr  & 2 yrs   & 3 yrs    & 4 yrs   & 1 yr   & 2 yrs  & 3 yrs  & 4 yrs\\ \midrule
1 &  84 & 29 & 11   & 2    & 14 & 5  & 2   & 0 \\ 
2  & 76 & 21 & 3     & -7   & 13 & 4  & 0   & -1 \\ 
3 &  48 & -7  & -25 & -35 & 8   & -1 & -4 & -6 \\
4 &  86 & 31 & 13  & 4     & 14 & 5  & 2  & 1 \\ 
5 &  88 & 33 & 15  & 6     & 15 & 6  & 2  & 1 \\ 
6 &  98 & 43 & 25  & 16   & 16 & 7  & 4  & 3 \\ \bottomrule
\end{tabular}
\label{tab:WQapsenario}
\end{table}%needs to be cleaned
	\begin{table}[htbp]
\centering
\caption{Necessary sites scenario for time variables}
\begin{tabular}{ccccc|ccccc}
\toprule
\multicolumn{1}{c}{} &  \multicolumn{4}{c}{ \#Samples required} & \multicolumn{4}{c}{\# sites required} \\ \cline{2-9}\noalign{\smallskip}
\multicolumn{1}{p{3cm}}{Elevation Bands}  & 1 yr   & 2 yrs  & 3 yrs     & 4 yrs   & 1 yr  & 2 yrs & 3 yrs  & 4 yrs \\ \midrule
1 &  51 & 13 & 0 & -7 & 9 & 2 & 0 & -1 \\ 
2 &  43 & 5 & -8 & -15 & 7 & 1 & -1 & -2 \\ 
3 &  15 & -24 & -36 & -43 & 3 & -4 & -6 & -7 \\ 
4 &  53 & 15 & 2 & -5 & 9 & 2 & 0 & -1 \\ 
5 &  55 & 17 & 4 & -3 & 9 & 3 & 1 & 0 \\ 
6 &  65 & 27 & 14 & 7 & 11 & 4 & 2 & 1 \\ \bottomrule
\end{tabular}
\label{tab:TVapsenario}
\end{table}
	\item The left side of both \autoref{tab:WQapsenario} and \autoref{tab:TVapsenario} show how many more samples are required to get a trend line with a power of .80. 
	\item  In \autoref{tab:WQapsenario} for elevation class 3, 48 more samples need to be collected if a trend line with a power of .80 is to be created after one year.  But if a trend line can wait to be created after two years, then there is a surplus of seven samples per year.  If four years can be waited there is a surplus of 35 samples which on the right side of the table translates into a surplus of 6 whole site locations per year.
	\item \autoref{tab:TVapsenario} works the same way as \autoref{tab:WQapsenario} but of course it uses different variables for the trend lines.
	\item results from previous draft
	\item any other papers like this?
\end{itemize}

\section{Discussion}

\subsection{Post hoc}
\begin{itemize}
	\item The results presented in \autoref{tab:SWposthoc} and \autoref{tab:TVposthoc} show how the calculated power is highly affected by number of observations more than anything else.
	\item In \autoref{tab:SWposthoc}, even when the r$^2$ and ES values are relatively low if the N is greater than 100 then the power is excellent.
	\item \autoref{tab:TVposthoc} show the effect of the ES on power.  Other than these lines being insignificant, many of the ES values are small according to Cohen and when compared to \autoref{tab:SWposthoc}.  Low ES values and low observations create low powers.  Low ES values com from low r$^2$ values.  The low r$^2$ values can be blamed for the insignificance of the lines and the poor powers. 
	\item Some lines are just not well described by Julian Date,$\sin$($\theta$), and $\cos$($\theta$) only.
\end{itemize}

\subsection{A priori}
\begin{itemize}
	\item How can these results be used?
	\item How can these results be manipulated?	
	\item The results in \autoref{tab:WQapsenario} and \autoref{tab:TVapsenario} can help with both of the problems of The park wanting a cheaper survey and researchers wanting more high elevation sites.
	\item The table can be used to re-organize sites across bands.
	\begin{itemize}
		\item In the current SS scheme there is a surplus of sites in lower elevation bands and a deficit for sites in higher elevations.
		\item Looking at the right side of \autoref{tab:WQapsenario}, if trends are desired after four years of data with a power of .80 and an ES of .15, seven sites may be taken from elevation bands 2 and 3 and 5 would need to be added to elevation bands 4,5, and 6.
		\item After this re-arrangement two sites may be completely discontinued.
		\item This saves time, effort, and money, but it is a very specific scenario.		
	\end{itemize}
	\item The downside of an a priori power analysis is that once you pick all the variables that go into it, you can' t change them in the future
	\begin{itemize}
		\item Variables that can change include how you divide the sites into elevation bands
		\item Trend line creation (alpha, variable selection)
		\item Power analysis ( power, and ES)
	\end{itemize}
	\item If during the hypothetical situation in which four years are waited to do another trend analysis, a better model is found, then the survey would need to be re-evaluated to reflect the new model.
	\begin{itemize}
		\item the model could require a different number of sites
	\end{itemize}
	\item Choices for power and ES could change
	\item planning with the a priori power analysis requires guessing the trends for the future.  % Thesis Conclusion?
	\item This guess will probably be based on the past , such as this one.
	\item This guess assumes that trends of the past will continue into the future
	\item The ANOVA/Bonferoni and the comparison between \citep{robinson2008ph} and the current trends shows that this is difficult.
	\item better understanding is needed
	\item At the end of the day the trends are positive!
\end{itemize}
	
