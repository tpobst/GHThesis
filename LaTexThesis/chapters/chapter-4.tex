\chapter{Power Analysis} \label{ch:poweranslysis}
\section{Methods}
\subsection{Introduction}
\begin{itemize}
	\item Statistics come with an inherent amount of error.
	\item The trend lines created in the trend analysis chapter have a defined error called type II error or $\beta$.
	%add a small table of error types
	\item $\beta$ describes failure to reject a false null hypothesis or failure to detect a trend in the data when there really is one.
	\item $\beta$ is usually described in terms of probability and it's opposite is called power(1-$\beta$)
	\item The power of a statistical test describes the probability that the test is true.
	\item The statistical test is the hypothesis test which tests if the coefficients of a regression line are zero.  So whether or not a trend exists.
	\item The power of the trend lines will state the "truth" of the slope of the trend.  A trend lilne with a power of 1.00 means that there is a 100$\%$ chance that the calculated slope is not zero.
	\item Using the earlier calculated trend lives as input, the power of each regression line was calculated with the help of G*power.  An a priori analysis was calculated to help determine the number of samples needed for desired levels of power.
\end{itemize}
\subsection{Body}
\begin{itemize}
	\item The objectives of the power analysis are to determine the power of the trend lines calculated from past observations and to determine an adaquate number of samples needed for different levels of power for the future.
	\item The inputs needed in the G*power program for a post hoc analysis are: number of observations (N), adjusted r$^2$, number of predictors, and Effect size.  N and adj.r$^2$ are  outputs from the trend analysis and effect size is calculated using G*power.  These values are reported in \autoref{tab:SWposthoc} and \autoref{tab:TVposthoc}.
	\item A post hoc analysis of the trend line data from \autoref{tab:TVposthoc} is not useful.  This is because most of the lines have terribly low r$^2$ values and are insignificant trend lines.  The power of an insignificant trend line is also insignificant.
	\item Post hoc analysis and a priori were  run on both methods for trend lines
	\item G*power is a free power analysis program written by four germen psychology professors.
	\item It runs the gamut in power analysis options and uses methods stated in \citep{cohen1992power}.
	\item G*power was used to calculate powers in the post hoc analysis and sample sizes for the a priori analysis.
	\item Excel was used along with results provided by G*power to create scenarios to finish up the a priori analysis.
\end{itemize}
\subsection{Procedures}
\subsubsection{Post hoc}
\begin{itemize}
	\item Data compiled in \autoref{tab:SWposthoc} and \autoref{tab:TVposthoc} give the inputs required for a post hoc power analysis on th previously created trend lines.
	\item required inputs for G*power include, ES(Effect Size), $\alpha$(alpha), number of observations, and number of predictors.
	\begin{itemize}
		\item ES is calculated in G*power by the Cohen method stated in \citep{cohen1992power} "A Power Primer".
		\item Alpha refers to the $\alpha$ of the trend lines (.05).
		\item Number of observations is given in trend line output from SPSS.
		\item Number of predictors is also stated in trend line output.
	\end{itemize}
	\item The calculate button will calculate the power
	\begin{itemize}
		\item This is all that is needed for a post hoc analysis.  It answers the question "What was the power of the survey ?" or "How strong are the trend lines that were computed?"
	\end{itemize}
	\item The calculated powers are reported along side their trend line inputs in \autoref{tab:SWposthoc} and \autoref{tab:TVposthoc}.
\end{itemize}
\subsubsection{A priori}
\begin{itemize}
	\item The a priori analysis can help survey planners to create a sampling survey that will produce trend lines with certain ES values and powers.
	\item There are two objectives to this analysis which are to create "power graphs", which are plots of power vs. sample size.  The other is to plan out an actual scenario for which samples can be added or subtracted to elevation bands for a desired power of .80 and an ES of .15.
	\item The power graphs are created in G*power using the "x-y plot for a range of values" button next to the "calculate" button.
	\item They-axis has the power values while the x-axis contains the number of observations or samples.  The power will increase with number of samples until it reaches 100.
	\item Four power graphs were created, one for each water quality variable.  If ES and power are set to .15 and .80 respectively for every presumed trend line then the only variable is number of predictors.  The number of predictors is set for each water quality variable (pH, ANC, NO$_3$,SO$_4$).  Taken from the earlier step-wise selection method (link to step-wise table).  Therefore only one "power graph" is needed for every trend line in each variable.
	\item ES and power can be chosen or kept constant based on reports by Cohen.
	\item Cohen's standardizations
	\item While the "power graphs" are useful in planning for the future of the stream survey, it can be shown that if ES and power are chosen, exact numbers of samples and sites can be added and subtracted from elevation bands.
	\begin{table}[htbp]
		\caption{A priori calculation in G*power when alpha, ES, and power are set to .05, .15, and .80 respectively.}
		\begin{center}
		\begin{tabular}{lcc}
		\hline\noalign{\smallskip}
		 & \multicolumn{1}{l}{Number of predictors} & N \\  \hline\noalign{\smallskip}
		pH & 6 & 98 \\ 
		ANC & 8 & 109 \\ 
		Nitrate & 8 & 109 \\ 
		Sulfate & 7 & 103 \\ 
		Time & 3 & 77 \\  \hline
		\end{tabular}
		\end{center}
		\label{APN}
		\end{table}%needs to be cleaned up
	\item The rest was done in excel
	\item This calculated number of observations can be divided by the number of samples collected in one year to get the number of years required to reach a power of .80.
	\item The analysis can be further conducted by calculating the number of samples per year to achieve a power of .80.  For this calculation all water quality variables were given the highest number of samples of 110 and 77 was used for the trends using only time variables.
		\begin{table}[htbp]
\caption{samples/year to achieve a power .80}
\begin{center}
\begin{tabular}{lrrrr}
\hline\noalign{\smallskip}
Years & 1 & 2 & 3 & 4 \\ \cline{2-5}\noalign{\smallskip}
Water Quality Variables & 110 & 55 & 37 & 28 \\ 
Time Variables & 77 & 39  & 26  & 19  \\ \hline\noalign{\smallskip}
\end{tabular}
\end{center}
\label{sytaapeighty}
\end{table}%needs to be cleaned up
	\item \autoref{sytaapeighty} is needed to calculate number of samples needed per elevation band to achieve a power of .80.  This number of samples can then be further divided to get a number of sites needed to achieve a power of .80.  If a trend line with a power of .80 is desired after one year ,for all water quality variables to be satisfied, 110 samples need to be collected.  If four years are waited then only 28 samples need to be collected per year.
	\item  To create this final table the number of samples per elevation  band was subtracted from the number of samples to achieve a power of .80 which gives us the number of samples needed in addition to what is currently collected to receive a power of .80.  These results are organized into samples needed per elevation band to achieve a power of .80 and seperated by years depending of how many years of data go into the trend lines.

\end{itemize}
\section{Results}
\subsection{Post hoc}
\begin{itemize}
	\item A post hoc power analysis was conducted for each of the two methods of trend analysis.
	\item \autoref{tab:SWposthoc} and \autoref{tab:TVposthoc} record the results of the post hoc analysis on the trend lines with variables created through the step-wise method and the trend lines created using only time variables respectively.  Included in these tables are the number of samples and r$^2$ variables from the trend analysis and effect size and power from the post hoc analysis.
	\item \autoref{tab:SWposthoc} and \autoref{tab:TVposthoc} are broken into the four analyzed water quality variables (pH, ANC, NO$_3$,SO$_4$) and divided into the tree time sets (93-02, 03-08, 09-12), and then further divided into the six elevation classes.
	\item use results from previous draft
	\item any similar power analysis?
\end{itemize}
\subsection{A priori}
\subsubsection{Power graphs}
\begin{itemize}
	\item The results of the a priori power analysis will be the most important for planning.
	\item The usual output is the "power graph" which plots power on the y-axis and total sample size on the x-axis.
	\item G*power outputs some very nice power graphs. The power graphs created from the a priori power analysis are presented in \autoref{fig:pHPowerGraph}, \autoref{fig:ANCnNPowerGraph}, \autoref{fig:SulfatePowerGraph}, and \autoref{fig:TVPowerGraph}.
	\item There were four power graphs created, three for the water quality variables and one for the time variables.  ANC and Nitrate both have the same number of predictors from the step-wise variable selection method and therefore create the same power graph. 
	\item each graph contains 3 lines representing 3 different ES choices: .15, .25, and .35.  These were chosen to mimic the choices of small, medium, and large effects standardized by Cohen in \citep{cohen1992power}.  Limitations of the G*power program left the best choices to be .15, .25, and .35.   A small effect of .02 was ignored because preliminary graph results showed it to be  not useful.	
\end{itemize}
\subsubsection{Planning with power analysis}
\begin{itemize}
	\item Using the ability of the a priori power analysis to compute a number of samples needed for a certain power, a scenario was played out to see how many sites needed to be added or could be removed from an elevation band in the stream survey.
	\begin{table}[htbp]
\caption{Years to acheive a power of .80}
\begin{tabular}{clccccc}
\toprule
\multicolumn{1}{p{2cm}}{Elevation Bands} & \multicolumn{1}{c}{Site \#} & \multicolumn{1}{p{2cm}}{Current n/yr} & pH &\multicolumn{1}{p{1cm}}{ ANC NO$_3$} & SO$_4$ & \multicolumn{1}{p{3cm}}{Time variables} \\  
\midrule
1 & 13 ,23, 24, 30, 479 & 26 & 3.77  & 4.19  & 3.96  & 2.96  \\ 
2 & \multicolumn{1}{p{4cm}}{4, 311, 268, 480, 310, 483, 147, 148, 484} & 34 & 2.88  & 3.21  & 3.03  & 2.26  \\ 
3 & \multicolumn{ 1}{p{4cm}}{114, 481, 482, 149, 66, 492, 137, 293, 270, 493, 485, 144, 224} & \multicolumn{ 1}{c}{62} & \multicolumn{ 1}{c}{1.58 } & \multicolumn{ 1}{c}{1.76 } & \multicolumn{ 1}{c}{1.66 } & \multicolumn{ 1}{c}{1.24} \\ 
4 & 143, 142, 73, 71 & 24 & 4.08  & 4.54  & 4.29  & 3.21  \\ 
5 & 74, 221, 251, 233 & 22 & 4.45  & 4.95  & 4.68  & 3.50  \\ 
6 & 253, 234 & 12 & 8.17  & 9.08  & 8.58  & 6.42  \\  
\bottomrule
\end{tabular}
\label{tab:currentyrsto.80}
\end{table}%needs to be cleaned
	\item This scenario was followed through with both methods of trend lines.
	\item \autoref{currentyrsto.80} records the six elevation bands along with the site numbers that belong to them.  In the column labeled ,current n per year, the amount of samples collected per elevation band in the year 2012 was tabulated.  The values in the remaining columns were calculated by dividing the number of samples given in \autoref{APN} by the current samples per year column in \autoref{currentyrsto.80}.
	\item Looking at the table there are 26  samples collected in elevation band one in one year.  In order to compute a trend line that receives a power of .80 with pH as the dependent  samples would need to be collected for 3.77 years before the trend line is computed.   The larges is elevation class for a trend line in ANC or NO$_3$ which requires 9.08 years.
	\begin{table}[htbp]
\caption{Necesary sites scenario for water quality variables}
\begin{tabular}{ccccc|ccccc}
\cline{2-9}
\multicolumn{1}{c}{} &   \multicolumn{4}{c}{ \#Samples required} & \multicolumn{4}{c}{\# sites required} \\ \cline{2-9}\noalign{\smallskip}
\multicolumn{1}{p{3cm}}{Elevation Bands}  & 1 yr  & 2 yrs   & 3 yrs    & 4 yrs   & 1 yr   & 2 yrs  & 3 yrs  & 4 yrs\\ \hline\noalign{\smallskip}
1 &  84 & 29 & 11   & 2    & 14 & 5  & 2   & 0 \\ 
2  & 76 & 21 & 3     & -7   & 13 & 4  & 0   & -1 \\ 
3 &  48 & -7  & -25 & -35 & 8   & -1 & -4 & -6 \\
4 &  86 & 31 & 13  & 4     & 14 & 5  & 2  & 1 \\ 
5 &  88 & 33 & 15  & 6     & 15 & 6  & 2  & 1 \\ 
6 &  98 & 43 & 25  & 16   & 16 & 7  & 4  & 3 \\ \hline
\end{tabular}
\label{tab:WQapsenario}
\end{table}%needs to be cleaned
	\begin{table}[htbp]
\centering
\caption{Necessary sites scenario for time variables}
\begin{tabular}{ccccc|ccccc}
\toprule
\multicolumn{1}{c}{} &  \multicolumn{4}{c}{ \#Samples required} & \multicolumn{4}{c}{\# sites required} \\ \cline{2-9}\noalign{\smallskip}
\multicolumn{1}{p{3cm}}{Elevation Bands}  & 1 yr   & 2 yrs  & 3 yrs     & 4 yrs   & 1 yr  & 2 yrs & 3 yrs  & 4 yrs \\ \midrule
1 &  51 & 13 & 0 & -7 & 9 & 2 & 0 & -1 \\ 
2 &  43 & 5 & -8 & -15 & 7 & 1 & -1 & -2 \\ 
3 &  15 & -24 & -36 & -43 & 3 & -4 & -6 & -7 \\ 
4 &  53 & 15 & 2 & -5 & 9 & 2 & 0 & -1 \\ 
5 &  55 & 17 & 4 & -3 & 9 & 3 & 1 & 0 \\ 
6 &  65 & 27 & 14 & 7 & 11 & 4 & 2 & 1 \\ \bottomrule
\end{tabular}
\label{tab:TVapsenario}
\end{table}
	\item The left side of both \autoref{WQapsenario} and \autoref{TVapsenario} show how many more samples are required to get a trend line with a power of .80. 
	\item  In \autoref{WQapsenario} for elevation class 3, 48 more samples need to be collected if a trend line with a power of .80 is to be created after one year.  But if a trend line can wait to be created after two years, then there is a surplus of seven samples per year.  If four years can be waited there is a surplus of 35 samples which on the right side of the table translates into a surplus of 6 whole site locations per year.
	\item \autoref{TVapsenario} works the same way as \autoref{WQapsenario} but of course it uses different variables for the trend lines.
	\item results from previous draft
	\item any other papers like this?
\end{itemize}
\section{Discussion}
\subsection{Post hoc}
\begin{itemize}
	\item The results presented in \autoref{tab:SWposthoc} and \autoref{tab:TVposthoc} show how the calculated power is highly affected by number of observations more than anything else.
	\item In \autoref{tab:SWposthoc}, even when the r$^2$ and ES values are relatively low if the N is greater than 100 then the power is excellent.
	\item \autoref{tab:TVposthoc} show the effect of the ES on power.  Other than these lines being insignificant, many of the ES values are small according to Cohen and when compared to \autoref{tab:SWposthoc}.  Low ES values and low observations create low powers.  Low ES values com from low r$^2$ values.  The low r$^2$ values can be blamed for the insignificance of the lines and the poor powers. 
	\item Some lines are just not well described by Julian Date,$\sin$($\theta$), and $\cos$($\theta$) only.
\end{itemize}
\subsection{A priori}
\begin{itemize}
	\item How can these results be used?
	\item How can these results be manipulated?	
	\item The results in \autoref{tab:WQapsenario} and \autoref{tab:TVapsenario} can help with both of the problems of The park wanting a cheaper survey and researchers wanting more high elevation sites.
	
	\item The table can be used to re-organize sites across bands.
	\begin{itemize}
		\item In the current SS scheme there is a surplus of sites in lower elevation bands and a deficit for sites in higher elevations.
		\item Looking at the right side of \autoref{tab:WQapsenario}, if trends are desired after four years of data with a power of .80 and an ES of .15, seven sites may be taken from elevation bands 2 and 3 and 5 would need to be added to elevation bands 4,5, and 6.
		\item After this re-arrangement two sites may be completely discontinued.
		\item This saves time, effort, and money, but it is a very specific scenario.		
	\end{itemize}
	\item The downside of an a priori power analysis is that once you pick all the variables that go into it, you can' t change them in the future
	\begin{itemize}
		\item Variables that can change include how you divide the sites into elevation bands
		\item Trend line creation (alpha, variable selection)
		\item Power analysis ( power, and ES)
	\end{itemize}
	\item If during the hypothetical situation in which four years are waited to do another trend analysis, a better model is found, then the survey would need to be re-evaluated to reflect the new model.
	\begin{itemize}
		\item the model could require a different number of sites
	\end{itemize}
	\item Choices for power and ES could change
	\item planning with the a priori power analysis requires guessing the trends for the future.  % Thesis Conclusion?
	\item This guess will probably be based on the past , such as this one.
	\item This guess assumes that trends of the past will continue into the future
	\item The ANOVA/Bonferoni and the comparison between \citep{robinson2008ph} and the current trends shows that this is difficult.
	\item better understanding is needed
	\item At the end of the day the trends are positive!
\end{itemize}
	