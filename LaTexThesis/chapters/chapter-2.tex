\chapter{Trend Analysis}\label{ch:TA}
\section{Methods}
\subsection{Introduction}

Trend analysis is a great way to characterize the park's water quality using data collected through the stream survey.  It is used to state the condition of the parks water bodies while trying to predict where the water quality is headed in the future.  A trend analysis on the stream survey data was conducted in 2002 and published in \citep{robinson2008ph} and then again in 2009 for the Biotics Effects report \citep{cai2012}.  This statistical procedure is used to discover sudden and gradual trends over time.  Of the ten elevation bands analyzed in \citep{robinson2008ph} six had negative julian date coefficients and the other four had no trend.  Of the 67 sites studied in the biotics effects report most showed no trend, 22 showed an increase in pH, and 2 showed a decrease\citep{cai2012}.  The trend analysis will use stream survey data from 1993 to 2012 using the statistical programs JMP and SPSS for analysis.

\subsection{Body}

Water quality is an ongoing concern for the park.  The acidification of the streams can have significant negative effects on wildlife and vegetation.   The stream survey collects water samples all over the park to monitor the health of the water.  The pH trends are used to indicate what condition the park is in.
	
Twenty years of data were available for this paper from the years 1993 to 2012.  The data used in these analyses are collected through the park wide stream survey.  Most samples are collected every two months and analyzed in a lab for many water quality variables including pH, ANC, nitrate, sulfate and some metals.  %description of lab process in Intro?
A single trend line containing all 20 years is unrealistic.  The difference in trends from \citet{robinson2008ph} and \citet{cai2012} suggests an inflection point in the trend line somewhere between 2002 and 2009.  For this reason and also to be able to easier compare results from this paper with those from \citet{robinson2008ph}, who used the years 1993 to 2002, a separate data set will be sectioned off from 1993 to 2002.  A third data set will be created after the year 2008 because this is the year that the Kingston and Bull run power plants installed scrubbers onto their stacks.  The hypothesis being the sulfate concentrations will be noticeably different and this difference could indicate a need for further study.  These three time sets will be analyzed separately. 

 Two more factor divisions of the data include dividing the data by elevation classes and and by four dependent variables (pH, ANC, Nitrate, and Sulfate).  The elevation classes used in this paper were set up to contain a minimum number of sites in order that the upper classes would not be too weak to be useful.  The divisions are presented here in \autoref{fig:ElevationBands}.
\begin{table}[htbp]
\caption{These elevation classes were created to add more weight to the higher elevations}
\begin{tabular}{clcp{5cm}}
\toprule
Elevation Classes & Meters (Feet)                              & n & Site \# \\ 
\midrule
1                        & 304.8-609.6 (1000-2000)           & 5   & 13 ,23, 24, 30, 479 \\ 
2                        & 609.6-762 (2000-2500)              & 9   & 4, 311, 268, 480, 310, 483, 147, 148, 484 \\ 
3                        & 762-914.4 (2500-3000)              & 13 & 114, 481, 482, 149, 66, 492, 137, 293, 270, 493, 485, 144, 224 \\ 
4                        & 914.4-1066.8 (3000-3500)         & 4   & 143, 142, 73, 71 \\ 
5                        & 1066.8-1371.6 (3500-4500)       & 4   & 74, 221, 251, 233 \\ 
6                        & $1371.6< (4500<)$                    & 2   & 253, 234 \\ 
\bottomrule
\end{tabular}
\label{tab:ElevationBands}
\end{table}
These are different from the historical eleven elevation bands which were separated by 500 foot intervals.  Some of the upper bands only contained only one site.  After years of collection this one site can describe its own characteristics but it cannot describe characteristics of the elevation band very well.  Elevation is an important factor in water quality and because the upper elevations are most effected by acid rain there needs to be enough sites in each band to make them statistically strong.  Without adding sites the best way to do this is to reorganize the elevation bands.  The dependent variables in this study are as mentioned before pH, ANC, Nitrate, and Sulfate.  %why these variables?( put into the intro)
Dividing all the data into three different time sets, six elevation bands and studying four different dependents will create 72 different trend lines.

    \subsubsection{Instruments}
    \begin{itemize}
    	\item done usisng statistical programs.
        \item Outliler determination and trend hypothesis.
        \begin{itemize}
        	\item Plot pH on y-axis vs. all time.  This visually represents that the slope does not equal zero.  Check outliers in this plot.  If they can be explained then fix or delete them. 
        	\item Plot pH on y-axis vs. elevation.  Visually check for trend of decreasing pH as elevation increases.
        	\item Plot pH vs. month.  To check for seasonality.
        	\item Outliers found include Abrams, Anakeesta sites, and storm flow.  Abrams is consistently found as an out lier within GRSM water quality projects using stream survey data for statistical purposes.  Abrams is located in the Cades Cove area of the park and sits in natural limestone bedrock.  This limestone increases the ANC of the streams so much that many of  the measured Abrams pH values are high enough to be outliers and are thrown out of the data.
        	\item Water quality at sites 237 and 252 are heavily influenced by Anakeesta geology introduced into the streams through road cuts.
        	\item Storm flow is also usually seen as an out lier in past GRSM water quality projects.  Storms can bring high intensity rain fall which can very quickly raise the levels of nitrate and sulfate pollution in the streams.  The runoff can also carry any pollutants that have come to rest on vegetation or the ground.  The lowered pH of the streams caused by the storm flow can cause leeching of the surrounding mineral geology in affected areas. Healthy streams can rebound to normal pH values, unhealthy streams can have lowered ANC due to the leaching.  Measurements taken from storm flow can show uncharacteristically low pH values and high amounts of metals.  In this way storm flow is sometimes considered an out lier.  Much of the water quality data has been characterized as base flow and storm flow by Dr.Cai, but not all it.  Water quality data after 2010 has not been characterized.  Dismissing all of  storm flow as an out lier is complicated by this lack of information.  Either; storm flow and base flow would need to be determined for the 2011 and 2012 data,all of the 2011 and 2012 data could be left out, or 2011 and 2012 would need to be characterized as base flow or storm flow.  Throwing out the years 2011 and 2012 would leave the last time set with only two years of data.  The data was compared with and without storm flow observations.  It was determined to manually select out lier storm flow observations.  They can be removed on a case by case basis during the regression procedure.
	\item review output  for normality, heteroscedasticity, cook's D, DFBETAS, DFFITS.
	\item Find proposed out lier in original data
	\item Justify its removal, remove it and run the regression method again
	\item The outputs will change every time an observation is removed.
	\end{itemize}
	\begin{table}[htbp]
\caption{Equations created through step-wise variable selection}
\begin{tabular}{lp{7.5cm}cc}
\toprule
Dependent (n)     &Model                                                                                                                                                                                                                                                                                                                                                                                                       & Adjusted $r^2$  & Model p \\ 
\midrule
pH (3116)            &$.673\times\log_2(\text{Sum Base Cations}) + (-.368\times \text{NO}_3) + (.262\times \text{Julian Day}) + (-.266\times \text{SO}_4) + (-.050\times\cos(\theta))$                                                                                                                                       & 0.630                  & $<$0.001 \\
ANC (3116)         &$ (.415\times \text{Sum Base Cations}) + (-.185\times \text{SO}_4) + (.595\times \text{Conductivity}) + (-.102\times \text{NO}_3) + (.019\times \text{Julian Date}) + (.005\times \text{Cl}) + (.005\times \sin(\theta))$                                                & 0.984                  & 0.049 \\
NO$_3$ (3116)   &$(-.295\times \text{SO}_4) + (-3.183\times \text{ANC}) + (2.19\times \text{Conductivity}) + ( .923\times \text{Sum Base Cations}) + (.120\times \text{Julian Date}) + (.051\times \text{Cl}) + (.047\times \sin(\theta)) + (.031\times \cos(\theta))$ &0.498                   & 0.017 \\ 
SO$_4$ (3116)   &$ (-.166\times \text{NO}_3) + (2.318\times \text{Conductivity}) + (-3.229\times \text{ANC}) + (1.033\times \text{Sum Base Cations}) + (.042\times \text{Julian Date})$                                                                                                                               & 0.720                  & $<$0.001 \\ 
\bottomrule
\end{tabular}
\label{tab:stepwiseeq}
\end{table}

	\item The variable selected through this process were used to create fixed models to be used while discovering the Julian Date coefficient for each water quality variable in each data set.
	\item If the step-wise equation had at least one time variable in it(Julian date, $\sin(\theta), \cos(\theta)$) then the rest were added.  This is presented in \autoref{variables}.
	\item along with the step-wise regression method, another regression analysis was done using only time based variables.  These are the Julian Date, $\sin(\theta)$, and $\cos(\theta)$ time variables.  This method was used to find trends in the water quality variables that are related to time only.		
        \item IBM's SPSS was used to conduct this trend analysis.
        \item These options were chosen for regression and assumptions for this procedure include.(from notebook and textbook)
\end{itemize}

\section{Results}
\begin{itemize}
	\item In \citep{robinson2008ph} table 4 reports julian date coefficients for four water quality variables (pH, ANC, NO$_3$, SO$_4$) by each elevation band.
	\begin{itemize}
		\item A similar layout was used in \autoref{tab:Step-wise julian date} and \autoref{tab:time vars}.
		\item This was done for continuity and ease of comparison.
	\end{itemize}
	\item The first trend analysis was completed using the step-wise method for choosing predictors and the results are presented in \autoref{tab:Step-wise julian date}
	\item The second trend analysis uses only time based variables as predictors and these results are presented in \autoref{tab:time vars}
	\item Both tables are modeled after table 4 in \citep{robinson2008ph}
	\item Each table is further divided into three different time sets: 1993-2002, 2003-2008, 2009-2012.
	\item Each of these time sets are further divided into six elevation bands
	\item Each of elevation band has the results of four trend lines, one for each of the studied water quality variables (pH, ANC, NO$_3$, SO$_4$).
	\item Each trend line is represented by its Julian date coefficient, the r$^2$ value for the trend line, and it's statistical significance.
	\item 2 of the 72 trend lines in \autoref{tab:Step-wise julian date} are insignificant.  In contrast only 20 of the trend lines in \autoref{tab:time vars} are significant.
	\item Insignificance is determined  by receiving a p-value greater than .05, the $\alpha$ of the trend line.  A p-value greater than .05 rejects the hypothesis that $\beta$ $\neq$ 0.  There is greater than a 5$\%$ chance that $\beta$=0.  There is to much chance of no trend line for the scientific community.
	\item Repeat trends from previous thesis
	\item Compare to \citep{robinson2008ph}, his trends are negative.
\end{itemize}
\section{Discussion}
\begin{itemize}
	\item Why were the trends insignificant?
	\item Why are the water quality trends trending the way they are at separate time sets ( discuss comparisons between sets in the ANOVA bonferoni section).
	\item How should the water quality variables have behaved based on known properties and \citep{robinson2008ph}.
	\item Very generally speaking these results are different than \citep{robinson2008ph} predicted.
	\item Water quality will continue to get better.
	\begin{itemize}
		\item because pollution is being regulated
	\end{itemize}
	\item there are still unknowns and prediction is still hard.
\end{itemize}
	
