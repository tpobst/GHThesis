\chapter*{Abstract}\label{ch:abstract}
The upper elevations of the GRSM receive some of the highest loading rates of acidifying nitrogen and sulfur species in North America \citep{johnson1992atmospheric}.  
Acid deposition will acidify the surface waters, which can harm anything that interacts with it, including the soils, life forms and streams.
The Great Smoky Mountains National Park (GRSM) is located in the southern Appalachians spanning eastern Tennessee and western North Carolina.
The GRSM is one the most visited parks in the U.S. and its conservation is a high priority for the National Park Services (NPS) which is tasked with preserving it.        
Park conservation is ever changing and  includes monitoring streams for the consequences of acid deposition.    
Stream grab samples from the park have been collected and studied under the park-wide stream survey program since 1993.
Stream health is determined by statistical analysis on water quality constituents such as pH, Acid Neutralizing Capacity (ANC), Nitrate, and Sulfate.
In 2002 \citet{robinson2008ph} found negative pH time trends, while more recently \citet{cai2013} reports increasing pH trends.
This paper will reproduce trends from the previous studies while adding trends from more recent data.
Along with the inflection of pH trends found by \citet{cai2013}  a pattern of decreasing sulfate concentrations found in the high elevation site Noland Divide may correspond to the recent decrease of sulfur dioxide emissions of the Kingston and Bull run power plants.
This may be studied by time trends, but a Bonferroni means comparison is used to further study the apparent correlation.
Currently elevation in the survey is characterized through the use of  elevation bands of which the sites belong to.
This is done to focus on the strong elevation trends of many of the water quality constituents.  %cite cai?
There is concern for the survey's lack of high elevation sites, where the park is affected by acid deposition the most \citep{weathers2006}.
This paper re-draws the boundaries of the elevation bands from eleven down to six in an attempt to strengthen the upper elevations.
A power analysis is then conducted to determine the inherent power of the bands to accurately report a time trend.

These analysis reported healthy streams with a continuing increase of pH and ANC trends while both the studied pollutants NO$_3^-$ and SO$_4^{2-}$ are decreasing.  
The concerns of lowering pH raised in \citet{robinson2008ph} are now not as important as those for SO$_4^{2-}$ desorption raised in \citet{cai2013}.  
The lack of elevation trend in SO$_4^{2-}$ was attributed to high elevation soil adsorption of depositional SO$_4^{2-}$ and a statement was made that SO$_4^{2-}$ remains absorbed to soil particles as long as soil water chemistry remains high in SO$_4^{2-}$ concentration and low in pH \citep{cai2011long}.  
The slope for the elevation trend of SO$_4^{2-}$ over the three sets is decreasing but most of the mean SO$_4^{2-}$ concentrations listed in \autoref{sec:DescriptiveStats} are increasing through time along with pH.
This suggests desorption of SO$_4^{2-}$ into the streams, thus raising the lower elevation SO$_4^{2-}$ concentrations up to the higher concentrations of upper elevation sites.
The Bonferroni analysis found no obvious connection between the decreases in power plant emissions and sulfate concentrations in the park.
It is possible that further study could produce more positive results of a delayed reaction from emissions to stream concentrations.
A post-hoc power analysis found all of the trends from the time trend analysis to be overpowered, this is probably due to the large datasets used.
While the a priori power analysis suggests about 110 observations are required for a trend to receive a power of .80.
This result is further manipulated to show how the sites in the Stream Survey may be re-organized for a more even elevational distribution of sites.


 




